\documentclass[a4paper]{article}
\usepackage{fullpage}
\usepackage{amsmath}
\usepackage{graphicx}
\usepackage[colorlinks]{hyperref}

\newcommand{\dd}{\text{d}}
\newcommand{\ii}{\text{i}}

\newcommand{\of}[1]{\left( {#1} \right)}
\newcommand{\fracd}[2]{\frac{\dd{}{#1}}{\dd{}{#2}}}

\newcommand{\dhref}[1]{\href{#1}{#1}}
\newcommand{\mailhref}[1]{\href{mailto:#1}{#1}}

\title{Final report on ``Heterogeneity of the Suprachiasmatic
Nucleus'' {\em BO 3612/2-1}}
\author{Hanspeter Herzel  \mailhref{h.herzel@biologie.hu-berlin.de},\\
        Grigory Bordyugov \mailhref{grigory.bordyugov@gmail.com}}
\date{\today}

\begin{document}
\maketitle

\begin{abstract}
  This DFG-supported project was aiming at uncovering the functional
  properties of circadian oscillator arrays of the SCN on different
  scales, ranging from single cell oscillations to ensemble effects.
  We report advances in several levels, including conceptual (phase of
  entrainment), single-cell mechanistic (molecular workings of
  regulatory gene networks), and tissue-wide (dynamics of oscillator
  ensembles). The common pattern appearing here is that a substantial
  number of circadian phenomena can be well understood in the
  framework of relatively simple mathematical models, parameterized by
  few parameters such as amplitude, period and phase of oscillations.
  Additionally, the collaboration with the lab of Prof Toru Takumi in
  RIKEN resulted in discovery of Choroid Plexus as another strong
  circadian oscillator in brain with the rhythmicity levels comparable
  with those of the SCN.
\end{abstract}

\tableofcontents

\section{Overview / Zusammenfassung}

\section{Main results: From modeling of single cell oscillations to
ensembles of neurons}
\subsection{Circadian entrainment phase: Towards a unified mathematical
model}

The perfect coordination of the oscillation periods between the
behaviour and the environment is not the only phenomenon in
chronobiology. Those periods can be perfectly aligned, but the
relative phase of this alignment, or the entrainment phase, is a no
less important feature of circadian entrainment and has been focus of
research for many decades, see, for
instance,~\cite{pittendrigh1981circadian}. Within this project, we
attempted to describe the behaviour of the entrainment phase in as
simple a mathematical model as possible and to uncover some common
patterns of its reaction to changes in the environment.

\subsubsection{Human chronotypes and the 180 degree rule}
A first striking example of the apparent unpredictability of
entrainment phase is the discrepancy between how precise our clocks
are in terms of the internal period and how broadly distributed are
our wake-up times. The reported precision of the clock is within a
quarter of hour, whereas the wake-up times (as a proxy of entrainment
phase) has a characteristic deviation of two hours across human
population \cite{duffy2005entrainment}. If our clocks are so precise,
why are our alarms not so?

We presented an attempt of clarification of this disproportion between
the precision of period and entrainment phase by introducing the
so-called 180-degree rule~\cite{granada2013human}. The rule asserts
that within a population with a however broad or narrow distribution
of internal periods $\tau$, distribution of phases of entrainment as
broad as 180 degrees are inevitable. Moreover, the width of the
entrainment range of the subject determines the sensitivity of the
entrainment phase to the mismatch between the Zeitgeber period $T$
(usually close to 24 hours) and the internal period $\tau$.

The explanation for the 180 degree rule is based on the inspection of
the structure of entrainment range in the simplest oscillator models.
The phase of entrainment assumes maximal and minimal values at the
borders of entrainment range and the those values span an interval of
180 degrees. Now if the oscillator is easily entrainable (or ``weak''
as we call it), it has a wide entrainment range in terms of tolerated
mismatches between $\tau$ and $T$ and, consequently, small changes in
$\tau-T$ (by changing $\tau$ for example across the population) are
translated into relatively small changes of entrainment phase. If, on
the other hand, the oscillator is ``strong'', i.e. it entrainment
range is narrow in terms of $\tau-T$, small changes in $\tau$ would be
translated in large changes of entrainment phase. Given the precision
of circadian rhythms in mammals, including humans, it is thus no
surprise that a narrow distribution of $\tau$ across the population
causes a broad distribution of entrainment phases aka wake-up times.

\begin{figure}
\begin{center}
\includegraphics[width=0.49\linewidth]{figures/phase/fig1A.pdf}
\includegraphics[width=0.49\linewidth]{figures/phase/fig1B.pdf}
\end{center}
\caption{
  {\bf A} Dependence of phase of entrainment within an Arnold tongue
  on Zeitgeber period $T$ and strength $Z_1$ for different
  photoperiods $\chi$.
  {\bf B} Dependence of phase of entrainment within an onion-like
  region on Zeitgeber period $T$ and photoperiod $\chi$.
\label{fig::phase}
}
\end{figure}

\subsubsection{Phase of entrainment in seasonality}
The phase of entrainment has been shown to be affected by the mismatch
between the internal clock period $\tau$ and the Zeitgeber period $T$.
Additionally, it is quite intuitive that a stronger Zeitgeber can
enforce its period $T$ on the internal clock even if the mismatch
between $\tau$ and $T$ is relatively large. Those facts results in
what is known as the Arnold tongue - a wedge-shaped entrainment
region, exemplarily shown in Figure~\ref{fig::phase} (A) with colours
encoding the phase of entrainment. We indeed see that the maximal
(yellowish) and the minimal (deep red) values of entrainment phases
are achieved close to the borders of the tongue.

But what if instead of changing Zeitgeber strength directly, we
consider changing photoperiod (the proportion of the light and dark
portion of the day) throughout the seasons? Intuitively we might
assume that a 12 hours day plus 12 hours night would make the
strongest Zeitgeber and shorter nights or shorter days would decrease
the effective Zeitgeber strength, even if the peak value of Zeitgeber
remains unchanged. Our investigation of the effect of seasonality on
the phase of entrainment resulted in an onion-formed counterpart of
the Arnold tongue in the ``period-photoperiod'' coordinates, see
Figure~\ref{fig::phase} (B). The main feature of the onion is the
dependence of phase of entrainment on the value of photoperiod. In
Figure~\ref{fig::phase} (A) by changing $Z_1$ we can achieve some
difference in phase of entrainment too, but moving along the vertical
axis in Figure~\ref{fig::phase} (B) produces more dramatic phase
changes.

As a matter of fact, it is possible to observe entrainment phases as
much as 180 degree apart by changing photoperiod only (consider the
cross section of the onion in Figure~\ref{fig::phase} at $T$ slightly
larger than 24 hours). This amount of phase change is impossible by
changing the Zeitgeber strength only. The reason for this sensitivity
is the observation that the onion is slightly skewed to the right: The
period in constant light condition is slightly larger than the period
in constant darkness condition (here chosen $\tau = 24$ hours). Thus
we have identified photoperiod as a major influencer on phase of
entrainment, which seems to have more potential to govern the phase
than the Zeitgeber strength only.


\subsection{Semi-automatic modeling for gene regulatory networks}
The common understanding of circadian rhythms is that they are mainly
produced on the transcriptional level by gene regulatory networks,
see~\cite{reppert2002coordination}. When building mathematical models
of such networks, one often stumbles over the large number of poorly
characterized kinetic parameters of the gene regulation and hence is
confronted with a problem of prescribing unknown values to the
numerous numerical parameters of the model.

Motivated by an earlier success of employing delay-differential
equations to model circadian gene regulatory networks
(see~\cite{korencic2012interplay}), we decided to make a further
attempt to make such modelling less supervised and more data-driven.

Within his masters dissertation, Matthew
Kondoff~\cite{kondoff2015modeling} looked at a way of automatically
generating computational models from human-readable description of
regulatory gene networks. His efforts resulted in an R package, with
the help of which a convenient way of numerical modeling was
providing, thus hiding the low-level nitty-gritty of translating the
network description into formulas and computer code.

\begin{figure}
\begin{center}
\includegraphics[width=\linewidth]{figures/matt/matt.pdf}
\end{center}
\caption{
  {\bf Upper panel} An example of a human-readable description of a
  gene regulatory network consisting of two genes (Bmal1 and
  Reverb$\alpha$) and three boxes (Ebox, RRE, Dbox).
  {\bf Lower panel} Comparison between oscillations in microarray
  data, the extracted first harmonic and the fit by a model with five
  genes.
\label{fig::matt}
}
\end{figure}

An example description of a simple two-gene, three-box network is
represented in Figure \ref{fig::matt}, upper panel. A highly
human-readable source code in R programming language automatically
produces then an optimized numerical simulation framework with
underlying low-level code for the nonlinear equations in C language
(for performance reasons). As a next step, the unknown parameters of
those nonlinear equations are found to fit the simulated oscillations
(Figure \ref{fig::matt}, lower panel, right plot) to given
experimental data (Figure \ref{fig::matt}, lower panel, left plot).
The data used here came from~\cite{zhang2014circadian}, being a set of
24 microarray assays collected in two hours intervals, totalling 48
hours worth of time series length. Before fitting, the parameters of
the $~$24 hours harmonic were extracted from the data (period,
relative phases and relative amplitudes) and used to define the cost
function in the fitting procedure. For the optimization procedure, the
so-called particle swarm optimzation (see~\cite{zambrano2012hydropso})
was used with optimized initial conditions,
see~\cite{richards2004choosing}.


\subsection{Moran's I as a measure for spatial (in)homogeneity between
oscillating cells}

In mammals, the SCN seems to play the central role in coordinating the
circadian rhythms across the whole
organism~\cite{reppert2002coordination}. The SCN is a densely packed
brain region containing dozens of thousands of rhythmic neurons and
can be anatomically classified in several regions, depending on what
neurotransmitter the underlying neurons can secret and react to. The
dorsal and ventral parts of the SCN represents the most obviously
different parts of the whole organ~\cite{yamaguchi2003synchronization}
and show quite different rhythmical behaviour. In this subproject, we
aimed at quantifying the spatial inhomogeneities across SCN slices in
terms of their synchronization properties.

The standard way to quantify the degree of synchronization in an
ensemble of oscillators has traditionally been the Kuramoto order
parameter $R$~\cite{kuramoto2012chemical}. The idea behind can be
visualized by placing each of the interacting oscillators on a unit
circle according to the oscillator's instantaneous phase and
calculating the centroid of resulting distribution of oscillators. The
absolute value of $R$, i.e. the distance of the centroid from the
origin, would then show the degree of instantaneous synchronization.
If, for instance, all oscillators are (nearly) uniformly distributed
along the unit circle, their centroid would be very close to origin
and $R$ would be close to zero. On contrary, if most of the
oscillators are grouped around a preferred phase, the centroid would
be close to that point and its distance from the origin (the absolute
value of $R$) would be close to one.

\begin{figure}
\begin{center}
\includegraphics[width=\linewidth]{figures/mi/mi.pdf}
\end{center}
\caption{
  {\bf (A)} and {\bf (C)} Colour-coded distribution of oscillation
  phases of neurons across an SCN slice under equinox and long-day
  conditions, respectively.
  {\bf (B)} and {\bf (D)} The corresponding distributions of phases.
  {\bf (E)} Time traces in the coordinate plane of Kuramoto's order
  parameter $R$ and Moran's I.
\label{fig::mi}
}
\end{figure}

This approach, however, assumes no special arrangement between the
oscillating units, such as, for example, their ordering in space
relatively to each other. Looking for a better way to measure such
spatial aspect of synchronization across neurons, Dr Christoph Schmal
stumbled upon the so-called Moran's I statistics, which has been
widely used in geographical and sociological sciences for quantifying
the spatial correlation in units with a certain arrangement in
space~\cite{moran1950notes}. Moran's I takes into account the
arrangement of units in space by calculating the correlation between
the units weighted by the distance between them. So, for example, a
black and white chess board would corresponding to a Moran's I value
of $-1$, reflecting the largest possible anti-correlation between
neighbouring units and a uniform homogeneous state would correspond to
the Moran's I value of 1 (the largest possible value). Moran's I
turned out to complement the traditional Kuramoto's order parameter
$R$ in a most insightful way when analyzing the rhythmicity of neurons
in SCN slices~\cite{schmal2017moran}. Figure~\ref{fig::mi} from that
paper shows two examples of phase distribution across SCN slices (see
(A) and (B) for colour-coded representations and (B) and (D) for the
probability distribution of phases).

When plotting the values of Moran's I vs Kuramoto's order parameter
$R$ (Figure~\ref{fig::mi} (E)), we have found scenarios with both
statistics giving complementary results. It is, for example, possible
that Moran's I remains close to zero, whereas the order parameter $R$
is non-zero. This situation would emerge when all neurons have phases
close to each other but the spatial correlation between neighbouring
neurons is absent. The opposite situation with relatively low values
of Kuramoto's order parameter $R$ and Moran's I close to one arises
when the spread of phases between the oscillators is large, but most
neighbours are relatively strongly correlated (green dots in
Figure~\ref{fig::mi} (E)). The influence of the photoperiod on the
observed interplay beween $R$ and Moran's I certainly deserves further
investigation, also in the light of the above discussion of the role
of photoperiod in entrainment phase.


\subsection{Robust circadian oscillations in the Choroid Plexus}

As the last piece of results within this project, we are reporting
here on the discovery of the Choroid Plexus (CP) as a extraordinary
robust circadian oscillator. By screening for periodic Bmal1-Eluc
expression across different brain regions, Dr Jihwan Myung found that
the periodicity of BMAL1 expression in the CP stood out among all
explants sampled~\cite{myung2017choroid}. The CP showed a remarkable
degree of phase synchrony on the single cell level, even beating that
of the SCN. Having assumed that the relative phasing of neurons across
the SCN can encode photoperiod, we speculated that the somewhat
simpler organization of the CP does not reflect such second-order
characteristics of circadian rhythmicity (such as, for instance, the
photoperiod) and hence adopts a less sophisticated, though stronger,
rhythmicity pattern.

The circadian oscillations in the CP were additionally modeled by the
so-called ``twist'' model. The twist represents the dependency of the
oscillator's period on its instantaneous amplitude. Such correlations
were found in the CP and the experimentally observed dependence of the
oscillation period on coupling, modulated by the gap junction blocker
meclofenamic acid (MFA), was successfully reproduced in a mathematical
model with twist. The main observed effect was the lengthening of the
oscillation period upon the application of MFA, which was mirrored by
the decreased coupling strength in the model.


\section{Formal}

\subsection{Published and submitted manuscripts}
A total of 7 manuscripts have been enabled by this project, five of
time being already published and two under review:
\cite{bordyugov2015tuning,schmal2015theoretical,kondoff2015modeling,schmal2017moran,wagner2017plant,myung2017choroid,schmal2017measuring}


\subsection{Personnel}
For the duration of the project, Dr Christoph Schmal had the position
of post-doctoral researcher at the ITB Berlin, funded by this project.

\subsection{Cooperation activities between Berlin and Tokyo}

\subsubsection{Telecom conferences}
During the three years of the project, we maintained bi-weekly Skype
conferences to synchronize our activities between Tokyo and Berlin.

\subsubsection{Visits}
The following visits between Toru Takumi's Lab at RIKEN and the
Institute for Theoretical Biology in Berlin were possible through the
project:
\begin{itemize}
  \item[-] Pia Rose visited Riken from Marth 2nd through April 25th 2015,
  \item[-] Christoph Schmal visited Riken from June 18th through July 3rd
  2014,
  \item[-] Hanpspeter Herzel visited Riken from July 17th through July
  20th 2014,
  \item[-] Toru Takumi visited the ITB in Berlin on three occasions from
  2014 through 2016,
  \item[-] GB visited RIKEN from February 10th through March 1st 2014.
\end{itemize}

\subsection{Conferences}
The results of the research on the project were presented at the
Gordon Conference on Chronobiology (Newport, RI in summer 2013, also
first contact with Jihwan Myung established there), at the conference
of the Japanese Society for Mathematical Biology (Osaka, Summer 2014),
at the congress of European Biological Rhythms Society in M\&unchen
(Fall 2014), and at the meeting of SRBR (summer 2016).

\subsection{Student theses and projects}
We are reporting the following supervising activity, made possible
within this project:

Matthew Kondoff build a tool for automatically generating and
exploring regulatory networks of genes and fitted them to microarray
data of Hogenesch et al., which resulted in an R package
\cite{kondoff2015modeling}. Anna-Marie Finger and Lorena Sofia Lopez
Zepeda investigated the circadian regulation of immune system and
liver. Sungsoo Lim and Marta del Olmo conducted computer-assisted
studies of entrainment phase in several models of circadian clock.


\bibliographystyle{vancouver}
\bibliography{jg-report}
\end{document}
