\documentclass[a4paper]{article}
\usepackage{fullpage}
\usepackage{amsmath}
\usepackage{graphicx}
\usepackage[colorlinks]{hyperref}

\newcommand{\dd}{\text{d}}
\newcommand{\ii}{\text{i}}

\newcommand{\of}[1]{\left( {#1} \right)}
\newcommand{\fracd}[2]{\frac{\dd{}{#1}}{\dd{}{#2}}}

\newcommand{\dhref}[1]{\href{#1}{#1}}
\newcommand{\mailhref}[1]{\href{mailto:#1}{#1}}

\title{Final report on ``Heterogeneity of the Suprachiasmatic
Nucleus'' {\em BO 3612/2-1}}
\author{Hanspeter Herzel  \mailhref{h.herzel@biologie.hu-berlin.de},\\
        Grigory Bordyugov \mailhref{grigory.bordyugov@gmail.com}}
\date{\today}

\begin{document}
\maketitle

\begin{abstract}
  This DFG-supported project was aiming at uncovering the functional
  properties of circadian oscillator arrays of the SCN on different
  scales, ranging from single cell oscillations to ensemble effects.
  We report advances in several levels, including conceptual (phase of
  entrainment), single-cell mechanistic (molecular workings of
  regulatory gene networks), and tissue-wide (dynamics of oscillator
  ensembles). The common pattern appearing here is that a substantial
  number of circadian phenomena can be well understood in the
  framework of relatively simple mathematical models, parameterized by
  few parameters such as amplitude, period and phase of oscillations.
  Additionally, the collaboration with the lab of Prof Toru Takumi in
  RIKEN resulted in discovery of Choroid Plexus as another strong
  circadian oscillator in brain with the rhythmicity levels comparable
  with those of the SCN.
\end{abstract}

\tableofcontents

\section{Overview and main results}

\section{Main results: From modeling of single cell oscillations to
ensembles of neurons}
\subsection{Semi-automatic modeling for gene regulatory networks}
The common understanding of circadian rhythms is that they are mainly
produced on the transcriptional level by gene regulatory networks,
see~\cite{reppert2002coordination}. When building mathematical models
of such networks, one often stumbles over the large number of poorly
characterized kinetic parameters of the gene regulation and hence is
confronted with a problem of prescribing unknown values to the
numerous numerical parameters of the model.

Motivated by an earlier success of employing delay-differential
equations to model circadian gene regulatory networks
(see~\cite{korencic2012interplay}), we decided to make a further
attempt to make such modelling less supervised and more data-driven.

Within his masters dissertation, Matthew
Kondoff~\cite{kondoff2015modeling} looked at a way of automatically
generating computational models from human-readable description of
regulatory gene networks. His efforts resulted in an R package, with
the help of which a convenient way of numerical modeling was
providing, thus hiding the low-level nitty-gritty of translating the
network description into formulas and computer code.

\begin{figure}
\begin{center}
\includegraphics[width=\linewidth]{figures/matt/matt.pdf}
\end{center}
\caption{
  {\bf Upper panel} An example of a human-readable description of a
  gene regulatory network consisting of two genes (Bmal1 and
  Reverb$\alpha$) and three boxes (Ebox, RRE, Dbox).
  {\bf Lower panel} Comparison between oscillations in microarray
  data, the extracted first harmonic and the fit by a model with five
  genes.
\label{fig::matt}
}
\end{figure}

An example description of a simple two-gene, three-box network is
represented in Figure \ref{fig::matt}, upper panel. A highly
human-readable source code in R programming language automatically
produces then an optimized numerical simulation framework with
underlying low-level code for the nonlinear equations in C language
(for performance reasons). As a next step, the unknown parameters of
those nonlinear equations are found to fit the simulated oscillations
(Figure \ref{fig::matt}, lower panel, right plot) to given
experimental data (Figure \ref{fig::matt}, lower panel, left plot).
The data used here came from~\cite{zhang2014circadian}, being a set of
24 microarray assays collected in two hours intervals, totalling 48
hours worth of time series length. Before fitting, the parameters of
the $~$24 hours harmonic were extracted from the data (period,
relative phases and relative amplitudes) and used to define the cost
function in the fitting procedure. For the optimization procedure, the
so-called particle swarm optimzation (see~\cite{zambrano2012hydropso})
was used with optimized initial conditions,
see~\cite{richards2004choosing}.


\section{Formal}

\subsection{Published and submitted papers}
papers
\cite{bordyugov2015tuning,schmal2015theoretical,kondoff2015modeling,schmal2017moran,wagner2017plant,myung2017choroid,schmal2017measuring}

\subsection{Student theses and projects}
Matthew Kondoff build a tool for automatically generating and
exploring regulatory networks of genes and fitted them to microarray
data of Hogenesch et al., which resulted in an R package
\cite{kondoff2015modeling}. Anna-Marie Finger and Lorena Sofia Lopez
Zepeda investigated the circadian regulation of immune system and
liver.  Sungsoo Lim and Marta del Olmo conducted computer-assisted
studies of entrainment phase in several models of circadian clock.



\section{Cooperation activities}

\subsection{Telecom conferences}
During the three years of the project, we maintained (mostly)
bi-weekly Skype conferences to synchronize our activities between
Tokyo and Berlin. The calls mainly included Christoph Schmal, Jihwan
Myung and GB.

\subsection{Visits between Japan and Germany}
The following visits between Tokyo and Berlin took place:
\begin{itemize}
  \item[-] Pia Rose visited Riken from Marth 2nd through April 25th 2015,
  \item[-] Christoph Schmal visited Riken from June 18th through July 3rd
  2014,
  \item[-] Hanpspeter Herzel visited Riken from July 17th through July
  20th 2014,
  \item[-] Toru Takumi visited the ITB in Berlin on three occasions from
  2014 through 2016,
  \item[-] GB visited Riken from February 10th through March 1st 2014.
\end{itemize}

\subsection{Conferences}
The results of the research on the project were presented at the
Gordon Conference on Chronobiology (Newport, RI in summer 2013, also
first contact with Jihwan Myung established there), at the conference
of the Japanese Society for Mathematical Biology (Osaka, Summer 2014),
at the congress of European Biological Rhythms Society in M\&unchen
(Fall 2014), and at the meeting of SRBR (summer 2016).

\bibliographystyle{vancouver}
\bibliography{jg-report}
\end{document}
