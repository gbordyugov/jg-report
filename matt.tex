\subsection{Semi-automatic modeling for gene regulatory networks} The
common understanding of circadian rhythms is that they are mainly
produced on the transcriptional level by gene regulatory networks,
see~\cite{reppert2002coordination}. When building mathematical models
of such networks, one often stumbles over the large number of poorly
characterized kinetic parameters of gene regulation (in particular,
the parameters of transcription, translation, and post-translational
modifications) and hence is confronted with the problem of prescribing
nearly unknown values to the numerous parameters of the model.
Motivated by an earlier success of employing delay-differential
equations to model circadian gene regulatory networks
(see~\cite{korencic2012interplay}), we decided to make a further
attempt to make such modelling less supervised and more data-driven.


\begin{figure}
\begin{center}
\includegraphics[width=\linewidth]{figures/matt/matt.pdf}
\end{center}
\caption{
  {\bf Upper panel} An example of a human-readable description of a
  gene regulatory network consisting of two genes (Bmal1 and
  Reverb$\alpha$) and three boxes (Ebox, RRE, Dbox).
  {\bf Lower panel} Comparison between oscillations in microarray
  data, the extracted first harmonic and the fit by a model with five
  genes.
\label{fig::matt}
}
\end{figure}

Within his masters dissertation completed at our lab at the ITB,
Matthew Kondoff~\cite{kondoff2015modeling} looked at a way of
automatically generating computational models from human-readable
description of regulatory gene networks. His efforts resulted in an R
package, with the help of which a convenient way of numerical modeling
of gene regulatory networks was provided, thus hiding the low-level
nitty-gritty of translating the network description into formulas and
computer code.


An example description of a simple two-gene, three-box network is
represented in Figure \ref{fig::matt}, upper panel. A highly
human-readable source code in R programming language automatically
produces an optimized numerical simulation scheme with underlying
low-level code for the nonlinear equations in C language (for
performance reasons). As a next step, the unknown parameters of those
nonlinear equations can be found to fit the simulated oscillations
(Figure \ref{fig::matt}, lower panel, right plot) to given
experimental data (Figure \ref{fig::matt}, lower panel, left plot).
The data used here came from~\cite{zhang2014circadian}, being a set of
24 microarray assays collected in two hours intervals, totalling 48
hours worth of time series length. Before fitting, the parameters of
the $~$24 hours harmonic were extracted from the data (period,
relative phases and relative amplitudes) and used to define the cost
function in the fitting procedure. For the optimization procedure, the
so-called particle swarm optimization
(see~\cite{zambrano2012hydropso}) was used with optimized initial
conditions, see~\cite{richards2004choosing}.

Despite the fact that DDE-based modeling drastically reduced the
number of parameters in the mathematical model, we still ended up with
some 30+ parameter for a model with five interacting genes. Given the
data we used and the fitting protocol (attempting to fit the period,
the relative phases together with amplitudes), the search for
parameters did not produced a unique set of parameters. This is to be
interpreted as a sign for possible redundancy of the chosen model.
