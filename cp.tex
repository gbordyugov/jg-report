\subsection{Robust circadian oscillations in the Choroid Plexus}

As the last piece of results within this project, we are reporting
here on the discovery of the Choroid Plexus (CP) as a extraordinary
robust circadian oscillator. By screening for periodic Bmal1-Eluc
expression across different brain regions, Dr Jihwan Myung found that
the periodicity of BMAL1 expression in the CP stood out among all
explants sampled~\cite{myung2017choroid}. The CP showed a remarkable
degree of phase synchrony on the single cell level, even beating that
of the SCN. Having assumed that the relative phasing of neurons across
the SCN can encode photoperiod, we speculated that the somewhat
simpler organization of the CP does not reflect such second-order
characteristics of circadian rhythmicity (such as, for instance, the
photoperiod) and hence adopts a less sophisticated, though stronger,
rhythmicity pattern.

The circadian oscillations in the CP were additionally modeled by the
so-called ``twist'' model. The twist represents the dependency of the
oscillator's period on its instantaneous amplitude. Such correlations
were found in the CP and the experimentally observed dependence of the
oscillation period on coupling, modulated by the gap junction blocker
meclofenamic acid (MFA), was successfully reproduced in a mathematical
model with twist. The main observed effect was the lengthening of the
oscillation period upon the application of MFA, which was mirrored by
the decreased coupling strength in the model.
