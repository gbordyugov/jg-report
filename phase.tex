\subsection{Circadian entrainment phase: Towards a unified mathematical
model}

The perfect coordination of the oscillation periods between the
behaviour and the environment is not the only phenomenon in
chronobiology. Those periods can be perfectly aligned, but the
relative phase of this alignment, or the entrainment phase, is a no
less important feature of circadian entrainment and has been focus of
research for many decades, see, for
instance,~\cite{pittendrigh1981circadian}. Within this project, we
attempted to describe the behaviour of the entrainment phase in as
simple a mathematical model as possible and to uncover some common
patterns of its reaction to changes in the environment.

\subsubsection{Human chronotypes and the 180 degree rule}
A first striking example of the apparent unpredictability of
entrainment phase is the discrepancy between how precise our clocks
are in terms of the internal period and how broadly distributed are
our wake-up times. The reported precision of the clock is within a
quarter of hour, whereas the wake-up times (as a proxy of entrainment
phase) has a characteristic deviation of two hours across human
population \cite{duffy2005entrainment}. If our clocks are so precise,
why are our alarms not so?

We explained this disproportion between the period and phase precision
by recognizing the so-called 180-degree rule. The rule asserts that
within a population with a however broad or narrow distribution of
internal periods $\tau$, distribution of phases of entrainment as
broad as 180 degrees are inevitable.


\subsubsection{Phase of entrainment in seasonality}
