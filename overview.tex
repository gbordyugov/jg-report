The main project aims were to understand the generation and
coordination of circadian rhythms in the mammalian SCN on different
scales which can range from the single cell periodicity to the
coordination of oscillations on the level of the whole organ. We are
glad to be able to report advances on at least three different levels
of this hierarchy.

On a single cell level, we present a computational framework for
mechanism-focused modeling of gene regulatory networks. Using a very
simple and intuitive description of the interaction between genes
together with their regulatory elements, the framework allows to
define interactions between the chosen genes and automatically tune
the parameters of the model, aiming at fitting a chosen set of
experimental data.  Whereas we have achieved a remarkable degree of
agreement between the simulated oscillations and their experimental
counterpart, the question of the uniqueness of the parameter values in
the model is still open and requires further investigations.

If one chooses to abstract away the details of how rhythms generated,
be it on the single cell or the organism level, and is only interested
in few fundamental oscillation properties such as amplitude, period
and the of the oscillations, the simplest mathematical oscillator
models turn out to be a great predictor for a substantial variety of
observed phenomena. In this project, we continued our previous efforts
to understand and describe the behaviour of entrainment phase under
different conditions. Our main contribution here is a new conceptual
treatment of photoperiod and its influence on entrainment phase. We
have discovered the ``entrainment onion'' \--- the counterpart of the
Arnold tongue under varying photoperiod (instead of Zeitgeber
strength) and showed that changes in photoperiod can effectuate
significantly larger deviations of entrainment phase.

Considering the SCN as an ensemble of interacting circadian neurons,
we proposed to use a new measure for its synchronization property -
the Moran's I statistics, previously nearly unknown on the field of
chronobiology. This statistics complements the more traditional
Kuramoto's order parameter in being sensitive not only to the global
distribution of the phases, but rather to how well neighbouring
oscillators are correlated. With the help of Moran's I, it became
possible to automatically classify between regimes in the SCN that
would have remained undistinguishable using Kuramoto's order parameter
only.

In conclusion, we report on the discovery of the extremely strong
rhythmicity in the choroid plexus (CP) (experiments performed in the
lab of Prof Toru Takumi). CP seems to be at least as robust an
oscillator as the SCN, having a higher amplitude of circadian
oscillations and a slightly shorter period. We proposed that the
remarkable rhythmic properties of the CP can be modelled by
oscillators with the so-called ``twist'', i.e. dependence of the
instantaneous period on the amplitude. Co-culture experiments with the
SCN and CP additionally suggest that the CP can tune the oscillations
in the SCN.
