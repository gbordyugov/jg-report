The main project aims were to understand the generation and
coordination of circadian rhythms in the mammalian SCN on different
scales, ranging from the single cell periodicity to the coordination
of oscillations on the level of the whole organ. We are glad to report
that we can report advances on at least three levels of this
hierarchy.

On a single cell level, we present a computational framework for
mechanism-focused modeling of gene regulatory networks. Using a very
simple and intuitive description of the interacting genes and their
regulatory elements as building blocks, the framework allows to define
interactions between the chosen genes and automatically tune the
parameters, aiming at fitting a chosen set of experimental data. Where
as we have achieved a remarkable degree of agreement between the
simulated oscillations and their experimental counterpart, the
question of the uniqueness of the parameter value in the model is
still open and requires further investigations.

If one chooses to abstract away the details of how rhythms generated,
be it on the single cell or organism level, and is only interested in
few fundamental oscillation properties such as amplitude, period and
the phase of the oscillations, the simplest mathematical oscillator
models turn out to be a great predictor for variety of observed
phenomena. In this project, we continued our previous efforts to
understand and describe the behaviour of entrainment phase under
different conditions. Our main contributions here is a new conceptual
treatment of the influence of photoperiod on entrainment phase. We
have discovered the ``entrainment onion'' - the counter part of the
Arnold tongue under varying photoperiod (instead of Zeitgeber
strength) and showed that changes in photoperiod can effectuate
significantly larger deviations of entrainment phase.
